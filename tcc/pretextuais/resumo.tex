%%%%%%%%%%%%%%%%%%%%%%%%%%%%%%%%%%%%%%%%%%%%%%%%%%%%%%%%%%%%%%%%%%%%%%%%%%%%%%%%
% resumo.tex
%
% Modelo de arquivo para uso com a classe uvvTeX2, para a formatação de
% trabalhos acadêmicos na Universidade Vila Velha (UVV) (https://www.uvv.br).
%
% Para maiores informações, visite:
%    https://github.com/uvv-computacao/uvvtex2
%
% Neste arquivo você deve escrever o resumo de sua monografia, ou seja, deve
% escrever o resumo em PORTUGUÊS. O resumo deve ser objetivo e listar os
% aspectos mais importantes de seu trablaho. Ao final do abstract, as
% palavras-chave que você definiu no arquivo principal da monografia serão
% inseridas automaticamente.
%%%%%%%%%%%%%%%%%%%%%%%%%%%%%%%%%%%%%%%%%%%%%%%%%%%%%%%%%%%%%%%%%%%%%%%%%%%%%%%%

% Não altere as linhas a seguir:
\setlength{\absparsep}{18pt}
\begin{resumo}

% Começe a escrever o abstract aqui:
Este trabalho apresenta um projeto voltado à construção de uma arquitetura em nuvem utilizando os serviços da Amazon Web Services (AWS), com foco em escalabilidade, segurança e otimização de custos. A proposta busca demonstrar como ambientes baseados em nuvem podem ser mais vantajosos que arquiteturas tradicionais on-premises, especialmente no que diz respeito à flexibilidade de recursos, agilidade na implantação e economia operacional. No entanto, também se reconhece que, embora a computação em nuvem ofereça diversos benefícios, sua eficácia depende diretamente da qualificação dos profissionais responsáveis pelo planejamento e implementação. A ausência de conhecimento técnico pode comprometer aspectos críticos como segurança, disponibilidade e custo, tornando a solução em nuvem até mesmo inferior a uma infraestrutura física bem gerenciada. O projeto inclui a definição e a automação de uma arquitetura baseada em três camadas: balanceamento, aplicação e banco de dados, implementada com uso de VPC, sub-redes públicas e privadas, ECS Fargate, RDS MySQL e S3, e provisionada via AWS CloudFormation. Os resultados demonstram que, quando bem configurada, a nuvem se mostra uma alternativa robusta, replicável e economicamente viável para organizações de diferentes portes.



% Não altere as linhas a seguir:
\textbf{Palavras-chave}: \imprimirpalavraschave.
\end{resumo}

